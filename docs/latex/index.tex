\begin{Desc}
\item[Autor:]John Edgar Congote Calle \end{Desc}
\begin{Desc}
\item[Versi\'{o}n:]0.3a1\end{Desc}
Genesting es un proyecto que intenta resolver el problema de nesting o anidamiento de figuras a trav\'{e}s del uso de algoritmos gen\'{e}ticos.\hypertarget{index_Introduction}{}\section{Introduction}\label{index_Introduction}
In the manufacturing industry the raw materials usually came in finite two-dimensional sheets, where the permanent goal is the reduction of waste materials. But there is a frequent problem: how to distribute two-dimensional patterns in a container sheet in order to get the maximum utilization of material? This is know as the Knapsack problem.

Nowdays many company make this job according with the empirical experience of their employers having two risks. The first one is that there is not way to know if their solution are going to be the best to minimize the amount of waste materials. The second risk is in the case of presuming that an employ has the optimal solution the knowledge would be in the hands of just one person or group of work and not as a part of a system or as a part of the company.\hypertarget{index_definition}{}\section{Definicion}\label{index_definition}
El problema de nesting se puede definir como encontrar una dispocision de patrones que esten dentro de otro de forma que se maximice el area utilizada. Este problema tiene muchas variantes, pero en el proyecto se trabajara especificamente sobre el caso de Knapsack.\hypertarget{index_objective}{}\section{Objetivo}\label{index_objective}
Encontrar una dispocicion de patrones que maximicen el area utilizada dentro de otro patron. Los patrones estan definidos como poligonos simples, los cuales pueden ser por definicion convexos o concavos pero solo pueden tener un adentro o mas claramente sus lineas no se intersectan.\hypertarget{index_case}{}\section{Caso de Estudio}\label{index_case}
Aunque el problema es teorico, el projecto quiere generar una aplicacion que pueda ser utilizada por la industria marroquinera, donde tienen que definir como distribuir los moldes de corte dentro de un cuero, en este caso el cuero se puede definir como el patron o poligono donde se tienen que colocar los demas patrones, y los moldes como los patrones internos. 